\documentclass[a4paper, 12pt]{article}
\usepackage{cmap} % Пакет для поиска в полученной пдфке 
\usepackage[utf8]{inputenc} % Ззамена кодировки файла на utf8 
\usepackage[T2A]{fontenc} % Подключение кодировки шрифтов 
\usepackage[russian]{babel} % Использование русского языка 
\usepackage[left=2cm, right=2cm, top=1cm, bottom=2cm]{geometry} % Изменение размеров полей 
\usepackage{indentfirst} % Красная строка в начале текста 
\usepackage{amsmath, amsfonts, amsthm, mathtools, amssymb, icomma, units, yfonts}
\usepackage{amsthm} % Пакет для нормального оформления теорем 
\usepackage{algorithmicx, algorithm}
\usepackage{algpseudocode}
\usepackage{graphicx}
\usepackage{tikz}
\usepackage{esvect}
\usepackage{enumitem}
\usepackage{dcolumn}
\usepackage{amsfonts}
\usetikzlibrary{calc,matrix}

\begin{document}
    \title{Математический анализ. Лекция 2}
    \date{08.09.2018}
    \maketitle

    \textit{Опр.} Множество отрезков $ \{[a_{1}, b_{1}], [a_{2}, b_{2}] ... \} $ называется \textbf{системой вложенных отрезков}, если $ [a_{n}, b_{n}] \supset [a_{n + 1}, b_{n + 1}] \forall n \in \mathbb{N}$, т.е. каждый следющий отрезок содержится в предыдущем.

    \textit{Лемма} о вложенных отрезках (\textit{принцип непрерывности Кантора}) Для каждой системы вложенных отрезков $ \exists c \in \mathbb{R} $, что $ \forall n \in \mathbb{N} : c \in [a_{n}, b_{n}] $\\
    \\
    $ \blacktriangleright A = \{a_{n} \;|\; n \in \mathbb{N}\}, B = \{b_{n} \;|\; n \in \mathbb{N}\}$\\
    Имеем $ \forall n, m \in \mathbb{N} : a_{n} \le a_{n + m} \le b_{n + m} \le b_{m}$\\
    Значит $\forall$ элемент из $A$ меньше $\forall$ элемента из $B$. По аксиоме непрерывности Кантора $ \exists c \in R : a_{n} \le c \le b_{n}, \; \forall a_{n} \in A; \forall b_{n} \in B \;\; \blacksquare $ \\

    \textit{Опр.} Система вложенных отрезков называется \textbf{стягивающейся}, если \\ $ \forall \epsilon > 0 \;\; \exists n \in \mathbb{N} : b_{n} - a_{n} < \epsilon $

    \textit{Теорема} Стягивающаяся система вложенных отрезков имеет ровно одну точку \\
    \\
    $ \blacktriangleright $ Предположим противное. $ c \ne c` \;\; \forall n \; c, c` \in [a_{n}, b_{n}] $ Для определённости $ c \le c` $ \\
    $ \forall n \; a_{n} \le c \le c` \le b_{n} \Rightarrow c` - c \le b_{n} - a_{n}  \Rightarrow $ \\
    $ \exists k \in \mathbb{N} : c` - c \le b_{k} - a_{k} \;\;$ противоречит определению $\blacksquare$\\

    \textit{Опр.} \textbf{Бесконечные десятичные дроби} - выражения вида $ \alpha = \pm \alpha_{0}\textbf{,}\alpha_{1}\alpha_{2}... \;$, где \\ $\alpha_{0} \in \{0, 1, 2, ...\} \\ \alpha_{n} \in \{0, 1, 3, ... , 9\} $\\
    Если знак "$+$" $\Rightarrow \alpha \ge 0 $ \\
    Если знак "$-$" $\Rightarrow \alpha \le 0 $ \\
    Сначала для б.д.д. определим операции "$\le$", "sup"{}, а затем "$+$"{}, "$-$"
    \\

    \textit{Опр.} Сравнение б.д.д.
    \begin{enumerate}
        \item Пусть $\alpha \ge 0, \beta \ge 0 \;\;$ б.д.д.
        $\alpha = \alpha_{0}\textbf{,}\alpha_{1}\alpha_{2}\alpha_{3}... \;\;$
        $\beta = \beta{0}\textbf{,}\beta{1}\beta{2}\beta{3}... $
        Скажем, что $ \alpha < \beta $, если волняется хотя бы одно утверждение
        \begin{itemize}
            \item $ \alpha_{0} < \beta_{0} $
            \item $ \alpha_{0} = \beta_{0} $ и $ \alpha_{1} < \beta_{1} $
            \item $ \alpha_{0} = \beta_{0} $ и $ \alpha_{1} = \beta_{1} $ и $ \alpha_{2} < \beta_{2} $
            \item ......
        \end{itemize}
        \item $ \alpha \le 0 $ и $ \beta \ge 0 $ и $\exists (\alpha$ или $ \beta) \ne 0 : \;\; \beta > \alpha$
        \item $ \alpha \le 0 $ и $ \beta \le 0 :\;\; \beta < \alpha \Leftrightarrow -\beta > -\alpha$
    \end{enumerate}
    \textit{Опр.} Определим точную верхнюю грань произвольного множества б.д.д\\
    \\
    Пусть X - ограниченное сверху множество б.д.д. (т.е. $\exists$ б.д.д. $c\;\;$ : $\;\; \forall x \in X \; : \; x \le c$)\\
    Определим $\gamma = \gamma_{0}\textbf{,}\gamma_{1}\gamma_{2} ... \; = sup X $ алгоритмом
    \\
    $ \gamma_{0} = max \{\alpha_{0} \;|\; \alpha \in X\} \;\;\;|\;\; A_{1} := \{\alpha \in X \;|\; \alpha_{0} = \gamma_{0}\}\\
    \gamma_{1} = max \{\alpha_{1} \;|\; \alpha \in A_{1}\} \;\;|\;\; A_{2} := \{\alpha \in A_{1} \;|\; \alpha_{1} = \gamma_{1}\}\\
    \gamma_{2} = max \{\alpha_{2} \;|\; \alpha \in A_{2}\} \;\;|\;\; A_{3} := \{\alpha \in A_{2} \;|\; \alpha_{2} = \gamma_{2}\}\\$

    \textit{Опр.} \textbf{Сумма} б. д. д.\\
    $\alpha = \alpha_{0}\textbf{,}\alpha_{1}\alpha_{2}\alpha_{3}... \;\;\\
    \beta = \beta{0}\textbf{,}\beta{1}\beta{2}\beta{3}...  \\
    (\alpha + \beta) := sup\{x + y \;|\;^{0 \le x \le \alpha}_{0 \le y \le \beta}\} \;\;\;\;$ x, y - конечные десятичные дроби\\

    Проверим \textit{аксиому непрервыности} (а. 14)\\
    $A$ и $B ,\; \forall a \in A ,\; b \in B ,\; a \le b ,\; \exists c \in R :\; a \le c \le b \;\;\;$ Проверим для $A$ и $B$ - б.д.д. \\
    Возьмём в множестве б.д.д $c := sup A\;$ Мы знаем (по построению) $\forall a \in A \;\; a \le c\;\;$
    % Проверим, что $	\forall b \in B \;\; b \ge c \; \blacktriangleright$ Допустим противное\\
    % $ \exists b \in B \;\; b < c; \;\; c = c_{0}\textbf{,}c_{1}c_{2}c_{3}... \;\; b = b_{0}\textbf{,}b_{1}b_{2}b_{3} ... $
    \\
    \textit{Теорема} \textbf{о единственности множества вещественных чисел}. \\
    Пусть $\mathbb{R}$ и $\widetilde{\mathbb{R}}$ - множества, удоволетворяющие всем аксиомам (1 - 14). Тогда имеет место \textit{биекция} $\mathbb{R} \longleftrightarrow \widetilde{\mathbb{R}}$ такая, что
    \begin{itemize}
        \item $ x + y \leftrightarrow \widetilde{x} + \widetilde{y} $
        \item $ xy \leftrightarrow \widetilde{x}\widetilde{y} $
        \item $ x \le y \leftrightarrow \widetilde{x} \le \widetilde{y} $
    \end{itemize}
\end{document}
